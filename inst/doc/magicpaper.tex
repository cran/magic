\documentclass[nojss]{jss}  
\usepackage{amssymb}

%%%%%%%%%%%%%%%%%%%%%%%%%%%%%%
%% declarations for jss.cls %%%%%%%%%%%%%%%%%%%%%%%%%%%%%%%%%%%%%%%%%%
%%%%%%%%%%%%%%%%%%%%%%%%%%%%%%


%% just as usual
\author{Robin K. S. Hankin}
\title{Recreational mathematics with \proglang{R}:
introducing the \pkg{magic} package}

%\VignetteIndexEntry{A vignette for the magic package}
%% for pretty printing and a nice hypersummary also set:
%% \Plainauthor{Achim Zeileis, Second Author} %% comma-separated
\Plaintitle{Recreational mathematics with R:
introducing the magic package}
\Shorttitle{Magic squares in R}

%% an abstract and keywords
\Abstract{
The \proglang{R} computer language~\citep{R} has been applied with a great deal
of success to a wide variety of statistical, physical, and medical
applications.  Here, I show that \proglang{R} is an equally superb research
tool in the field of recreational mathematics.


An earlier version of this vignette was published as~\citet{hankin2005}. 
}

\Keywords{Magic squares}
% -*- mode: noweb; noweb-default-code-mode: R-mode; -*-

\Address{
Robin K. S. Hankin\\
University of Cambridge\\
19 Silver Street\\
Cambridge \\
CB3 9EP\\
E-mail: \email{rksh1@cam.ac.uk}
}


%% need no \usepackage{Sweave.sty}

\begin{document}

\section{Overview}
Recreational mathematics is easier to recognize than define, but seems
to be characterized by requiring a bare minimum of ``raw material'':
complex notation is not needed, and problems are readily communicated
to the general public.  

This is not to say that all problems of recreational mathematics are
trivial: one could argue that much number theory is recreational in
nature; yet attempts to prove Fermat's Last Theorem, or the search for
ever higher perfect numbers, have been the catalyst for the
development of many fruitful new areas of mathematics.  

The study of magic squares is also an example of nontrivial
recreational mathematics as the basic concept is simple to grasp---yet
there remain unsolved problems in the field whose study has revealed
deep mathematical truths.

Here, I introduce the ``magic'' package, and show that \proglang{R} is an
excellent environment for the creation and investigation of magic
squares.  I also show that one's appreciation of magic squares may be
enhanced through computer tools such as \proglang{R}, and that the act of
translating `paper' algorithms of the literature into \proglang{R} idiom can
lead to new insight.


\section{Introduction}
Magic squares have essentially zero practical use; their
fascination---like much of pure mathematics---lies in the appeal of
\ae sthetics and structure rather than immediate usefulness.

The following definitions are almost universal:
\begin{itemize}
\item A {\em semimagic square} is one all of whose row sums equal all its
columnwise sums (i.e. the magic constant).

\item A {\em magic square} is a semimagic square with the sum of both
unbroken diagonals equal to the magic constant.

\item A {\em panmagic square} is a magic square all of whose broken
diagonals sum to the magic constant.

\end{itemize}
(all squares are understood to be $n\times n$ and to be {\em
normal\/}, that is, to comprise $n^2$ consecutive
integers\footnote{Most workers require the entries to start at 1,
which is the convention here; but there are several instances where
starting at~0 is far more convenient.  In any case, if \code{x} is
magic, then \code{x+n} is magic for any integer \code{n}.}).  Functions
\code{is.semimagic()}, \code{is.magic()}, and \code{is.panmagic()}
test for these properties.

\begin{Schunk}
\begin{Sinput}
> require(magic)
\end{Sinput}
\end{Schunk}

A good place to start is the simplest---and by far the most commonly
encountered---magic square, {\em lo zhu}:

\begin{Schunk}
\begin{Sinput}
> magic(3)
\end{Sinput}
\begin{Soutput}
     [,1] [,2] [,3]
[1,]    2    7    6
[2,]    9    5    1
[3,]    4    3    8
\end{Soutput}
\end{Schunk}

This magic square has been known since antiquity (legend has it that
the square was revealed to humanity inscribed upon the shell of a
divine turtle).  More generally, if consecutive numbers of a magic
square are joined by lines, a pleasing image is often obtained
(figure~\ref{magic7}, for example, shows a magic square of order~7;
when viewed in this way, the algorithm for creating such a square
should be immediately obvious).

\begin{figure}[htbp]
  \begin{center}
\begin{Schunk}
\begin{Sinput}
> magicplot(magic.2np1(3))
\end{Sinput}
\end{Schunk}
\includegraphics{magicpaper-004}
    \caption{Magic square of order~7\label{magic7} in graphical form
(obtained by \texttt{magicplot(magic.2np1(3))}) }
  \end{center}
\end{figure}

Function \code{magic()} takes an integer argument~$n$ and returns a
normal magic square of size $n\times n$.  There are eight equivalent
forms for {\em lo zhu\/} or indeed any magic square, achieved by
rotating and reflecting the matrix~\citep{benson1976}; such equivalence
is tested by \code{eq()} or \code{\%eq\%}.  Of these eight forms, a
magic square \code{a} is said to be in {\em Fr\'{e}nicle's standard
form} if \code{a[1,1]}$\leq$\code{b[1,1]} whenever \code{a \%eq\% b},
and \code{a[1,2]<a[2,1]}.  Function \code{is.standard()} tests for
this, and function \code{as.standard()} places a magic square in
standard form.  Magic squares returned by \code{magic()} are always in
standard form.

A typical (paper) algorithm for placing magic square \code{a} in
standard form would be ``rotate \code{a} until
\code{a[1,1]<min(a[1,n],a[n,1],a[n,n])} then, if \code{a[1,2]>a[2,1]},
take the transpose''.  I shall show later that expressing such an
algorithm in \proglang{R} leads to new insight when considering magic
hypercubes.

A wide variety of algorithms exists for calculating magic squares.
For a given order $n$, these algorithms generally depend on $n$
modulo~4.

A typical paper algorithm for magic squares of order~$n=4m$ would go as
follows.
\begin{quote}
Algorithm 1:
in a square of order~$4m$, shade the long major diagonal.  Then shade
all major diagonals distant by a multiple of~4 cells from the long
diagonal.  Do the same with the minor diagonals.  Then, starting with
``1'' at the top left corner and proceeding from left to right and top
to bottom, count from~1 to $n^2$, filling in the shaded squares
with the appropriate number and omitting  the unshaded ones
[figure~\ref{magicsquare8.halfdone}].  Fill in the remaining
(unshaded) squares in the same way, starting at the lower right
corner, moving leftwards and upwards [figure~\ref{magicsquare8}].
\end{quote}

Such paper algorithms are common in the literature but translating
this one into code that uses \proglang{R}'s vectorized tools effectively can
lead to new insight.  The magicness of such squares may be proved by
considering the increasing and decreasing sequences separately.

\begin{figure}[htb]
  \begin{center}
\includegraphics{magicpaper-005}
    \caption{Half-completed magic square of
        order\label{magicsquare8.halfdone} 8}
  \end{center}
\end{figure}

\begin{figure}[htb]
  \begin{center}
\includegraphics{magicpaper-006}
    \caption{Magic square of order\label{magicsquare8} 8}
  \end{center}
\end{figure}

The interesting part of the above paper algorithm lies in determining
the pattern of shaded and unshaded squares\footnote{If \code{a <-
matrix(1:(n*n),n,n)}, with \code{jj} a Boolean vector of length~$n^2$
with \code{TRUE} corresponding to shaded squares, then with it is
clear that \code{a[jj] <- rev(a[jj])} will return the above magic
square.}.  As the reader may care to verify, parsing the algorithm
into \proglang{R} idiom is not straightforward.  An alternative, readily
computed in \proglang{R}, would be to recognize that the repeating $4\times 4$
cell \code{a[2:5,2:5]} is \code{kronecker(diag(2),matrix(1,2,2)) -> b}
say, replicate it with \code{kronecker(matrix(1,3,3),b) -> g}; then
trim off the border by selecting only the middle elements, in this
case \code{g[2:9,2:9]}.  Function \code{magic.4n()} implements the
algorithm for general $m$.

\section{Magic hypercubes}

One of the great strengths of \proglang{R} is its ability to handle arbitrary
dimensioned arrays in an efficient and elegant manner.

Generalizing magic squares to magic hypercubes~\citep{hendricks1973} is
thus natural when working in \proglang{R}.  The following definitions represent
a general consensus, but are far from universal:

\begin{itemize}
\item A {\em semimagic hypercube} has all ``rook's move'' sums equal
to the magic constant (that is, each~$\sum_{i_r=1}^n
a[i_1,i_2,\ldots,i_{r-1},i_r,i_{r+1},\ldots,i_d]$ with $1\leqslant
r\leqslant d$ is equal to the magic constant for all values of the
other i's).
\item A {\em magic hypercube} is a semimagic hypercube with the
 additional requirement that all $2^{d-1}$ long (ie extreme
 point-to-extreme point) diagonals sum correctly.
\item A {\em perfect magic hypercube} is a magic
hypercube with all nonbroken diagonals summing correctly\footnote{This
  condition is quite restrictive; in the case
of a tesseract, this would include subsets such
as $\sum_{i=1}^na[1,i,n-i+1,n]$ summing correctly.}.
\item A {\em pandiagonal hypercube} is a perfect magic hypercube with
all broken diagonals summing correctly.
\end{itemize}
(a magic hypercube is understood to be of dimension \code{rep(n,d)}
and normal).  Functions \code{is.semimagichypercube()},
\code{is.magichypercube()} and \code{is.perfect(a)} test for the first
three properties; the fourth is not yet implemented.  Function
\code{is.diagonally.correct()} tests for correct summation of the
$2^d$ (sic) long diagonals.

\subsection[Magic hypercubes of order 4n]{Magic hypercubes of
  order~{\boldmath $4n$}}

Consider algorithm 1 generalized to a $d$-dimensional hypercube.  The
appropriate generalization of the repeating cell of the $8\times 8$
magic square discussed above is not immediately obvious when
considering figure~\ref{magicsquare8.halfdone}, but the \proglang{R} formalism
(viz \code{kronecker(diag(2),matrix(1,2,2))}) makes it clear that
the appropriate generalization is to replace \code{matrix(1,2,2)} with
\code{array(1,rep(2,d))}.

The appropriate generalization for \code{diag(2)} (call it \code{g})
is not so straightforward, but one might be guided by the following
requirements:
\begin{itemize}
\item The dimension of \code{g} must match the first argument to
  \code{kronecker()}, viz \code{rep(2,d)}
\item The number of 0s must be equal to the number of 1s:
\code{sum(g==1)==sum(g==0)}
\item The observation that \code{diag(2)} is equal to its transpose
would generalize to requiring that \code{aperm(g,K)} be
identical to \code{g} for any permutation \code{K}.
\end{itemize}
These lead to specifying that \code{g[i1,...,id]} should be zero if
$(i_1,\ldots,i_d)$ contains an odd number of 2s and one otherwise.

One appropriate \proglang{R} idiom would be to define a function
\code{dimension(a,p)} to be an integer matrix with the same dimensions
as \code{a}, with element $(n_1,n_2, ..., n_d)$ being $n_p$, then if
$\mbox{\code{jj}}=\sum_{i=1}^d\mbox{\code{dimension(a,i)}}$, we can specify
\code{g=jj*0} and then \code{g[jj\%\%2==1] <- 1}.

Another application of \code{kronecker()} gives a hypercube that is of
extent $4m+2$ in each of its \code{d} dimensions, and this may be
trimmed off as above to give an array of dimensions \code{rep(4m,d)}
using \code{do.call()} and \code{[<-}.  The numbers may be filled in
exactly as for the 2d case.

The resulting hypercube is magic, in the sense defined
above\footnote{If I had a rigorous proof of this, the margin might be
too narrow for it.}, although it is not perfect; function
\code{magichypercube.4n()} implements the algorithm.  The ability to
generate magic hypercubes of arbitrary dimension greater than one is
apparently novel.

\subsubsection{Standard form for hypercubes}
Consider again the paper definition for Fr\'{e}nicle's standard form
of a magic square \code{a}: it is rotated so that the smallest number
appears at the top left; then if \code{a[1,2]<a[2,1]}, the transpose
is taken.

When coding up such an algorithm in \proglang{R} with an eye to generalizing it
to arbitrarily high dimensional hypercubes, it becomes apparent that
``rotation'' is not a natural operation.  The generalization used in
the package is directly suggested by \proglang{R}'s array capabilities: it is
a two-step process in which the first step is to maneuver the smallest
possible element to position \code{[1,1,...,1]} using only operations
that reverse the index of some (or all) dimensions.  An example would
be \code{a <- a[1:n,n:1,1:n,n:1]}.

The second step is to use function \code{aperm()} to ensure that the
appropriate generalization of \code{a[1,2]<a[2,1]}, which would be
\begin{eqnarray*}
\mbox{\code{a[1,1,...,1,2]}}<\mbox{\code{a[1,...,2,1]}}<\ldots\nonumber\\
\ldots<\mbox{\code{a[1,2,...,1]}}<\mbox{\code{a[2,1,...,1]}}
\end{eqnarray*}
holds; the appropriate \proglang{R} idiom is
\code{a <- aperm(a,order(-a[1+diag(d)])))}.

This generalization of Fr\'{e}nicle's standard form to arbitrary
dimensional hypercubes appears to be new; it arises directly from the
power and elegance of \proglang{R}'s array handling techniques.

\section{Conclusions}

The \proglang{R} language is a natural environment for the investigation of
magic squares and hypercubes; and the discipline of translating
published algorithms into \proglang{R} idiom can yield new insight.  These
insights include a new generalization of Fr\'{e}nicle's standard form
to hypercubes, and also what appears to be the first algorithm for
generating magic hypercubes of any dimension,

Insofar as magic squares and hypercubes are worthy of attention, it is
worth creating fast, efficient routines to carry out the ``paper''
algorithms of the literature.  I hope that the \code{magic} package
will continue to facilitate the study of these fascinating objects.

\subsubsection*{Acknowledgements}
I would like to acknowledge the many stimulating and helpful comments
made by the \proglang{R}-help list over the years.

\bibliography{magic}
\end{document}
